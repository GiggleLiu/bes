\documentclass[prb,twocolumn,showpacs,superscriptaddress]{revtex4-1}
\usepackage{graphicx}
\usepackage{xcolor}
\usepackage{amsmath}
\usepackage{amssymb}
\usepackage{epsfig}
\usepackage{bm}
\usepackage[normalem]{ulem}
\usepackage[colorlinks,linkcolor=blue,anchorcolor=blue,citecolor=blue,urlcolor=blue]{hyperref}
\renewcommand{\v}[1]{{\bf #1}}
\newcommand{\Eq}[1]{Eq.~\ref{#1}}
\newcommand{\Fig}[1]{Fig.\ref{#1}}
\newcommand{\Ref}[1]{Ref.\onlinecite{#1}}
\newcommand{\Tr}{{\rm Tr}}
\newcommand{\Det}{{\rm Det}}
\renewcommand{\Im}{{\rm Im}}
\renewcommand{\Re}{{\rm Re}}
%\newcommand{\ch}{{\rm ch}}
%\newcommand{\sh}{{\rm sh}}


\usepackage{color}
\newcommand {\B}{\textcolor {blue}}
\newcommand {\R}{\textcolor {red}}
\newcommand{\Comment}[1]{\textcolor{red}{#1}}
\newcommand{\Insert}[1]{\textcolor{blue}{#1}}
\newcommand{\tk}{\tilde{\mathbf{k}}}
\renewcommand{\k}{\mathbf{k}}
\newcommand{\tb}{\tilde{\mathbf{b}}}
\newcommand{\bv}{\mathbf{b}}
\newcommand{\x}{\mathbf{x}}
\newcommand{\tx}{\tilde{\mathbf{x}}}
\newcommand{\sig}{\mathbf{\sigma}}
% Journal reference.  Comma sets off: name, vol, page, year
%\def\journal #1, #2, #3, 1#4#5#6{{\sl #1~}{\bf #2}, #3 (1#4#5#6) }
%\def\pr{\journal Phys. Rev., }
%\def\prb{\journal Phys. Rev. B, }
%\def\prl{\journal Phys. Rev. Lett., }
%\def\pl{\journal Phys. Lett., }
%\def\np{\journal Nucl. Phys., }

\begin{document}

\title{Numerical Methods for Bulk Entanglement Spectrum}

\author{Jin-Guo Liu}
\affiliation{National Laboratory of Solid State Microstructures $\&$ School of Physics, Nanjing
University, Nanjing, 210093, China}

%\date{\today}

%\pacs{72.15.Qm, 72.10.Fk, 74.20.-z, 71.27.+a}
%
%74.20.-z  Theories and models of superconducting state
%64.60.ae  Renormalization-group theory

\maketitle

\section{Deduction of Tracing Procedure}

\begin{figure}
\includegraphics[width=8.5cm]{lattice.eps}
\caption{The lattice configuration, the lattice sites in A are colored in black while those in B in red.
$\x$ in blue is used to label the position of blocks(inter-block label). The block spacing is $L\times L$ and the size of A-block is $S\times S$.}
\label{lattice}
\end{figure}

This note is an improved deduction of the main result in \Ref{Hsieh2014}.
The target is to get the bilinear expectation value $\langle c^\dag_{\k\sig} c_{\k\sig'}\rangle$ for region A in Fig.\ref{lattice},
with $\k$ the k-vector for block position $\x$ and $\sig$ is the inna-block index
(it is different with the k-vector $\tk$ which is for site position $\tx$).

The first step is to express $c_{\k,\sig}$ in terms of $c_{\tk}$.
Here, the inna-block label $\sig$ is also the relative site position with respect to $\x$(the number of possible values of $\sig$ is $S\times S$).
In other words, $\x+\sig$ is the global position of a site($\tx$).

With $N_0$ the number of sites and $N(=N_0/L^2)$ the number of blocks, we have
\begin{align}
    c_{\k,\sig}&=\frac{1}{\sqrt{N}}\sum\limits_{\x}e^{-i\k\x}c_{\x\sig}\\
                &=\frac{1}{\sqrt{NN_0}}\sum\limits_{\x}e^{-i\k\x}\left(\sum\limits_{\tk}e^{i\tk(\sig+\x)}c_{\tk}\right)\\
                &=\frac{1}{\sqrt{NN_0}}\sum\limits_{\tk}\left(\sum\limits_{\x}e^{i(\tk-\k)\x}\right)e^{i\tk \sig}c_{\tk}
\end{align}
Up to now, the deduction is straight forward, then we are going to to evaluate the equation in the parentheses.

Soon we notice that only those $\tk$s 'equivalent' to $\k$ will contribute to the above equation, that is
\begin{align}
    c_{\k,\sig}=\sqrt{\frac{N}{N_0}}\sum\limits_{\tk}\delta(\tk-\k-n_1\bv_1-n_2\bv_2)e^{i\tk \sig}c_{\tk}
\end{align}
with $n_1,n_2$ arbituary integers.

To find out these equivalent points, take a glance of the first Bruillouin zones as shown in Fig.\ref{bzone},
\begin{figure}
\includegraphics[width=8.5cm]{bzone.eps}
\caption{The first Brillouin zones for site space(with unit vector $\tb$) and block space(with unit vector $\bv$),
momentum equivalent to $\Gamma$ with respect to $\bv$ are colored in red, the collection of these points is $\Gamma^*$.}
\label{bzone}
\end{figure}

The number of members in $\Gamma^*$ is $2L\times L$ and they are $n_1\bv_1+n_2\bv_2$.
Thus we have,
\begin{align}
    c_{\k,\sig}=\sqrt{\frac{N}{N_0}}\sum\limits_{\tk-\k\in\Gamma^*}e^{i\tk \sig}c_{\tk}
\end{align}

Then, it is straight forward to express a bilinear operator after the above tracing procedure,
\begin{align}
    \langle c^\dag_{\k\sig} c_{\k\sig'}\rangle=\frac{N}{N_0}\sum\limits_{\tk-\k\in\Gamma^*}e^{i\tk(\sig'-\sig)}\langle c_{\tk}^\dag c_{\tk}\rangle
\end{align}

\bibliography{ref}

\end{document}
